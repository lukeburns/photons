\documentclass{birkjour}

\begin{document}

\title{Spinors are for photons too}

%----------Author 1
\author{Lucas Burns} 
\address{
135 W Lorain St\\
OCMR 488\\
Oberlin, OH 44074 \\
USA}
\email{lukemburns@gmail.com}

\thanks{Thanks to Rob Owen and Dan Styer.}

%----------classification, keywords, date
% \subjclass{Primary 99Z99; Secondary 00A00} 

\keywords{Spinors, Photons, Electrodynamics, Quantum Mechanics, Spacetime}

\date{Deadline: 4 January 2016}

\begin{abstract}
  Spinors are most frequently used to describe spin-$\frac{1}{2}$ particles. Here we construct a spinor which describes the electrodynamics and quantum mechanics of a photon in the 3-dimensional geometric algebra $G_3$. We then introduce the wave equation $\nabla \psi = p s \psi$, to which this spinor is a solution, where $p$ is the energy-momentum and $s$ is the spin of the photon. This paper aims to demonstrate that spinors can be used to describe spin-1 particles too and to raise questions regarding the connection between the geometry of spinors and the spin of particles.
\end{abstract}

\maketitle

\section{Introduction}

Geometric algebra endows spinors with a simple geometric interpretation, as rotation and dilation operators. While spinors are most frequently used to describe spin-$\frac{1}{2}$ particles, it is a straightforward task to use them to describe the simplest of spin-1 particles: the photon.

Photons are massless, spin-1 particles whose energy $E = \hbar \omega$ and momentum $\vec p = \hbar \vec k$ are equal in magnitude and determined by the angular frequency $\omega$ and wave vector $\vec k$, which describe the kinematics of circularly polarized electromagnetic waves.

Here we show that the currents of spinor-valued solutions to the second-order wave equation satisfy the homogeneous Maxwell equations and are identified with the electric and magnetic fields of a photon. The familiar energy and momentum operators $i \hbar \partial_t$ and $-i \hbar \vec \nabla$ are presented and consolidated into the odd-valued operator $\nabla = e^\mu \partial_\mu$, with which we introduce the first-order spinor wave equation $\nabla \psi = p s \psi.$ We conclude with comments and questions about this equation and the connection between spinors and the spin of particles.

\section{Spacetime in $G_3$}

Throughout this paper, we work with the orthonormal basis $\{e_1, e_2, e_3\}$ and use the notation 

\begin{align}
  e_{\mu \nu ... \gamma} &:= e_\mu e_\nu ... e_\gamma\\
  e_0 &:= e_{123}\\ 
  e^\mu &:= e_\mu^{-1} 
\end{align}

for $\mu, \nu, \gamma = 0,1,2,3$. 

Odd valued multivectors in $G_3$ take the place of four-vectors in Minkowski space, and the scalar product is used as their inner product. We let $x, k, \nabla \in G_3^-$ be odd valued, where

\begin{align}
  x &:= e_\mu x_\mu = t e_0 + \vec x \\
  k &:= e_\mu k_\mu = \omega e_0 + \vec k \\
  \nabla &:= e^\mu \partial_\mu = e^0 \partial_t + \vec \nabla
\end{align}

which satisfy

\begin{align}
  x * x &= \vec x^2 - t^2\\
  k*x &= \vec k \cdot \vec x - \omega t\\ 
  \nabla x &= 4 \\
  \nabla k * x &= k,
\end{align}

where the scalar product $A * B := \left< AB \right>_0$ is the scalar part of the product between multivectors $A, B \in G_3$.

Choosing even valued multivectors for $x, k, \nabla$ instead amounts to choosing the opposite metric convention. We also use natural units $c = \hbar = 1$. Any undefined and unlabeled symbol can be taken to be a real-valued scalar.

\section{The Wave Equation}

Let $\psi : G_3^- \to G_3^+$. The wave equation

\begin{equation}
  \nabla * \nabla \psi(x) = (\nabla^2 - \partial_t^2) \psi(x) = 0,
\end{equation}

has solutions of the form

\begin{equation}
  \psi(x) = \rho_1 e^{i k * x} + \rho_2 e^{-i k * x}, \label{eq:super}
\end{equation}

where $i$ is an arbitrary bivector, and the invariant $k*k = \vec k^2 - \omega^2 = 0.$

\section{Electric and Magnetic Fields}

Suppose $i = e_{12}$ and $\hat k = e_3$. If $\rho_1 = 1$ and $\rho_2 = 0$, then 

\begin{equation}
  \psi(x) = e^{i k * x} = e^{\hat k e_0(\vec k \cdot \vec x - \omega t)} \label{eq:right}
\end{equation} 

describes a right circularly polarized electromagnetic wave propagating in the $\hat k$ direction. 

The electric and magnetic fields of the photon are identified as the currents

\begin{align}
  \vec E &:= e_1 \psi = \cos(k*x) e_1 + \sin(k*x) e_2\\ 
  \vec B &:= e_2 \psi = \cos(k*x) e_2 - \sin(k*x) e_1,
\end{align}

which satisfy the homogeneous Maxwell equations.

The one-sided multiplication above rotates $\vec E$ and $\vec B$ with the correct frequency, but to make sense of the general action of $\psi$ on an arbitrary vector, we must make use of the two sided multiplication

\begin{equation}
  (\psi^{1/2})^\dagger e_\mu \psi^{1/2}.\footnote{Use of the square root of $\psi$ ensures the currents rotate with the proper frequency.}\label{eq:currents}
\end{equation}

Note that $e_0$ and $e_3$ are unchanged under this action. The currents $\vec E$ and $\vec B$ fully describe the kinematics of the spinor $\psi$.

If we had considered $\rho_1 = 0$ and $\rho_2 = 1$, then 

\begin{equation}
  \psi(x) = e^{-i k * x} = e^{-\hat k e_0 (\vec k \cdot \vec x - \omega t)} \label{eq:left}
\end{equation}

describes a left circularly polarized electromagnetic wave propagating in the $\hat k$ direction. 

Hence, Equation \ref{eq:super} is a linear superposition of right and left circularly polarized electromagnetic waves.

\section{Energy, Momentum, and Spin}
\label{sec:qm}

The energy of the right-circularly polarized photon in Equation \ref{eq:right} is given by

\begin{equation} 
  i \hbar \partial_t \psi = E \psi, \label{eq:en}
\end{equation} 

and the momentum is given by 

\begin{equation}
  -i \hbar \vec \nabla \psi = \vec p \psi. \label{eq:mom}
\end{equation}

These operators are familiar but are dependent on the spin plane $i$ of the photon. If we had instead looked at the left handed photon given by Equation \ref{eq:left}, then we would have unconventional energy and momentum operators given by 

\begin{equation} 
  -i \hbar \partial_t \psi = E \psi \label{eq:unen}
\end{equation} 

and

\begin{equation}
  i \hbar \vec \nabla \psi = \vec p \psi, \label{eq:unmom}
\end{equation}

due to the fact that the spin plane of this photon is $-i$. The seemingly negative energies given by the operator $i \hbar \partial_t$ in Equation \ref{eq:unen} are actually positive energies of a left-circularly polarized photon (if $i$ is right-handed). 

We can write this all down more compactly and clearly as a first order wave equation. With $\nabla = e^\mu \partial_\mu$ and $\hbar = 1$,

\begin{equation}
  \nabla \psi = \nabla (\pm i \, k * x) \psi = p s \psi,\label{eq:mod}
\end{equation}

where $p = \hbar k$ is the energy-momentum spinor satisfying $p * p = E^2 - \vec p^2 = 0$, and $s = \pm i \hbar$ encodes the spin of the photon.

The spin and energy-momentum of photons are tightly intertwined quantities. For this reason, we break a tradition followed by most relativistic wave equations and move the spin $\pm i \hbar$ from the operator side to the eigenvalue side of the equation.

\section{A Note on the Dirac Equation}

It is worthwhile to briefly point to some similarities and differences between this equation and the Dirac equation, written in Hestenes's spacetime algebra,\cite{hestenes}

\begin{equation}
  \nabla \psi i \sigma_3 = m \psi \gamma_0.
\end{equation}

A key similarity is that they can both be well described as first-order, eigenvalue equations of energy-momentum operators. Two key differences are (1) the difference in dimensionality of the algebra in which the spinors live, which may place kinematical limitations on Equation \ref{eq:mod} and (2) the eigenvalue in Equation \ref{eq:mod} is spinor-valued, whereas the eigenvalue in the Dirac equation is scalar valued, which may place a limitation on the Dirac equation.

Further investigation of the solutions of Equation \ref{eq:mod} will occur in a future paper. In particular, are there solutions which describe spin-$\frac{1}{2}$ particles? If so, how are spin-$\frac{1}{2}$ solutions related to spin-1 solutions?

\section{Conclusion}

That photons, spin-1 particles, can be so simply described by spinors is not surprising given the geometric interpretation of spinors, and it begs the question, what then characterizes spin-$\frac{1}{2}$ particles, if not their mathematical representation as spinors?

While the factor of a $\frac{1}{2}$ appearing in Equation \ref{eq:currents} is indicative of the nature of spinors, its appearance is an entirely mathematical feature and has no bearing on the physics of the photon described. In particular, we are only forced to write down $\psi^{1/2}$ if we are dealing with rotations of vectors orthogonal to the spin plane.

This is highly suggestive of when a factor of $\frac{1}{2}$ might appear as a physical consequence. If a spinor encodes rotation in more than one plane, then we are forced to compute $\psi^{1/2}$ to find the currents, unlike the case of the electromagnetic fields above, but still, this factor of $\frac{1}{2}$ is a mathematical choice. The connection between orthogonality and spin will be explored in depth in a following paper.

\begin{thebibliography}{9}

\bibitem{hestenes}
  David Hestenes,
  \emph{Zitterbewegung in Quantum Mechanics},
   Arizona State University, Arizona
  1994.

\end{thebibliography}

\end{document}
